\def\year{2018}\relax
%File: formatting-instruction.tex
\documentclass[letterpaper]{article} %DO NOT CHANGE THIS
\usepackage{aaai18}  %Required
\usepackage{times}  %Required
\usepackage{helvet}  %Required
\usepackage{courier}  %Required
\usepackage{url}  %Required
\usepackage{graphicx}  %Required
\frenchspacing  %Required
\setlength{\pdfpagewidth}{8.5in}  %Required
\setlength{\pdfpageheight}{11in}  %Required

%%%% My packages
\usepackage[utf8]{inputenc}

\newcommand{\citet}[1]
{\citeauthor{#1} \shortcite{#1}}
\newcommand{\citep}{\cite}
\newcommand{\citealp}[1]
{\citeauthor{#1} ̃\citeyear{#1}}

\graphicspath{{Figures/}}
\usepackage{tikz,filecontents}
\usetikzlibrary{shapes,arrows,shadings,patterns}
\usepackage{pgfplots}
\pgfplotsset{compat=newest}
\pgfplotsset{plot coordinates/math parser=false}
\newlength\figureheight
\newlength\figurewidth
\usepackage{epstopdf}
\usepackage{gensymb}
\usepackage{float}
\usepackage{amsfonts}
\usepackage[cmex10]{amsmath}
\usepackage{multirow}
\usepackage{todonotes}
\usepackage{colortbl}
\usepackage[ruled,noend]{algorithm2e}

% Examples of several macros
\newcommand*{\SET}[1]{\ensuremath{\boldsymbol{#1}}}
\newcommand*{\VEC}[1]{\ensuremath{\boldsymbol{\mathrm{#1}}}}
\newcommand*{\FAM}[1]{\ensuremath{\mathrm{#1}}}
\newcommand*{\MAT}[1]{\ensuremath{\boldsymbol{\mathrm{#1}}}}
\newcommand*{\OP}[1]{\ensuremath{\mathrm{#1}}}
\newcommand*{\NORM}[1]{\ensuremath{\left\|#1\right\|}}
\newcommand*{\DPR}[2]{\ensuremath{\left \langle #1,#2 \right \rangle}}
% Markings
\newcommand{\bsy}[1]{\boldsymbol{#1}}
\newcommand{\bma}[1]{\mathbf{#1}}
\newcommand{\ovl}[1]{\overline{#1}}
\renewcommand{\hat}[1]{\widehat{#1}}
\def\D{\mathrm{d}} % define \D as beeing staight d
\def\T{\mathrm{T}} % define \T as beeing staight T
\newcommand{\e}[1]{\times 10^{#1}} %define scientific notation

\newtheorem{theorem}{Theorem}

\newcommand{\alert}[1]{\textcolor{red}{#1}}
\usepackage[caption=false,font=footnotesize]{subfig}
\usepackage{url}
\makeatletter
\newcommand{\removelatexerror}{\let\@latex@error\@gobble}
\makeatother


% correct bad hyphenation here
\hyphenation{op-tical net-works semi-conduc-tor}


%PDF Info Is Required:
  \pdfinfo{
/Title (myICAPS_PaperAppendix)
/Author (Paul Barde)}
\setcounter{secnumdepth}{0}  
\begin{document}
% The file aaai.sty is the style file for AAAI Press 
% proceedings, working notes, and technical reports.
%
\title{Parallel MCTS-Pseudo Code}


% for over three affiliations, or if they all won't fit within the width
% of the page, use this alternative format:
% 
\author{ }

% make the title area
\maketitle

% As a general rule, do not put math, special symbols or citations
% in the abstract
\section*{Objects}
\subsection*{General objects}
\begin{description}
\item[Action space $A$]: tuple with action space of a node. 

\item[Histogram $\bsy{h}$]: a histogram $\bsy{h}$ with $n_b$ bins is an array of int of size $n_b$ and $\bsy{h}[i]$ is the number of elements in the $i^{th}$ bin. 

\item[Modification sequence $\bsy{m}$]: log generated whenever a WorkerTree expands a new node. It is an array composed of $[id,h,h_p,o,\Delta]$. With $id$ the attribute of the WorkerTree that expanded the new node. $h,h_p,o$ the attributes of the corresponding MasterNode. And $\Delta$ the return obtained from expanding this new node.

\item[Action Dict $\bsy{D}$]: a dictionary giving for each action the corresponding index : $\{0\degree : 0, 45\degree:1,...,315\degree : 7\}$
\end{description}

\subsection*{Worker objects}
\begin{description}
\item[WorkerTree $\bsy{\theta}$]: a worker tree is a tree operating a MCTS-UCT search for a given weather scenario. It has the following attributes: 
\begin{itemize}
    \item \textbf{the root node $\bsy{\nu_0}$}: a reference pointing towards the workernode type root node. 
    \item \textbf{a simulatior $S$}: a simulator with initial state and given weather conditions. 
    \item \textbf{the time horizon $T$}: final time of the simulator (and horizon of the search).
    \item \textbf{estimated time $T_{min}$}: time estimated at the initialisation of the search to reach the destination given the weather conditions. 
    \item \textbf{worker index $id$}: an int that characterises the weather scenario on which the worker is searching. 
    \item \textbf{a buffer $\bsy{B}$}: a chronological list of the modification sequences that have not been transmitted to the master yet. 
\end{itemize}

\item[WorkerNode $\bsy{\nu}$]: How nodes are represented in the Workers domain. It has the following attributes:  
\begin{itemize}
    \item \textbf{a parent $p$}: a reference toward the parent node.
    \item \textbf{origins $\bsy{\omega}$}: a list of the actions taken form the root node to get to this node. We call arm the last action taken $o$. 
    \item \textbf{children $\bsy{c}$}: a list of references towards the children nodes.
    \item \textbf{actions $\bsy{a}$}: a list of the remaining available actions.
    \item \textbf{values $\bsy{Q}$}: a array of size len($A$) containing Histograms. $\bsy{Q}$[i] is the histogram of the rewards provided by children which origin is $A$[i]. 
\end{itemize}
\end{description}

\subsection*{Master objects}
\begin{description}
\item[MasterTree $\bsy{\Theta}$]: Master tree that manages $n_s$ different WorkerTree working in parallel. Each WorkerTree is searching on a different weather scenario. 
\begin{itemize}
    \item \textbf{nodes $\bsy{N}$}: a dictionary with key the hash of a node and value a reference toward the corresponding MasterNode. 
    \item \textbf{proba $\bsy{P}$}: a array with the probability of occurrence of each scenario. 
\end{itemize}

\item[MasterNode $\bsy{\mu}$]: How nodes are represented in the Master domain. It has the following attributes:  
\begin{itemize}
    \item \textbf{a hash $h$}: hash based on the origins $\bsy{\omega}$ of the corresponding WorkerNode $\bsy{\nu}$.
    \item \textbf{arm $o$}: action taken from the parent node to extend the present node. 
    \item \textbf{parent hash $h_p$}: hash of the parent. 
    \item \textbf{rewards $\bsy{R}$}: an array of size (number of scenario, len($A$)) containing Histograms of returns. Thus for each MasterNode $\bsy{\mu}$ we have for each action taken from it and for each scenario a Histogram of all the obtained returns. 
\end{itemize}
\end{description}



\removelatexerror
\begin{algorithm}[H]
\SetAlgoLined
\SetKwProg{Fn}{function}{:}{}
\SetKwFor{For}{for}{:}{}
%\SetKwIF{If}{if}{:}{\textbf{•}
%\SetKwWhile{While}{while}{:}{}
\Fn{\textsc{UctSearch}($\bsy{s_0}$)}{
create root node $\nu_0$ with state $\bsy{s_0}$\\
 \While{\emph{within computational budget}}{
 	$\nu_l, \, \bsy{s} \leftarrow$ \textsc{TreePolicy}$(\nu_0$, $\bsy{s_0})$\\
 	$\Delta \leftarrow$ \textsc{DefaultPolicy}($\nu_l$, $\bsy{s}$)\\
 	$\nu.\bsy{Q}[:]$ $\leftarrow$ $\Delta$\\
 	\textsc{Backup}($\nu_l,\Delta$)\\
 	$h, \, h_p \leftarrow$ \textsc{GetHash}($\nu_l$), \textsc{GetHash}($p(\nu_l)$)\\
 	$\bsy{m}$ $\leftarrow$ $[id,h,h_p,\bsy{\omega}(\nu_l)[-1],\Delta]$\\
 	append $\bsy{m}$ to $\bsy{B}$\\
 	\If{it is time to feed master}{
 	send $\bsy{B}$ to master\\
 	$\bsy{B}$ $\leftarrow$ []}
 	}
	}{ \textit{ }}
	%\vspace{1cm}
	\\
	\vspace{0.5cm}
\Fn{\textsc{TreePolicy}($\nu$, $\bsy{s}$)}{
	\While{$\bsy{s}$ \emph{is nonterminal}}{
	\If{$\nu$ \emph{not fully expanded}}{
		\textbf{return} \textsc{Expand}($\nu$, $\bsy{s}$)}
	\Else{$\nu\leftarrow$ \textsc{BestChild}($\nu,Cp,\rho,\bsy{\Theta}$)\\
			$\bsy{s}\leftarrow S(\bsy{s},a(\nu))$}}
	}{\textit{ } \\ \textbf{return} $\nu, \, \bsy{s}$}
    \\
	\vspace{0.5cm}
\Fn{\textsc{Expand}($\nu$, $\bsy{s}$)}{
	choose random $a \in \bsy{a}(\nu)$ the untried actions of $\nu$\\
	add a new child $\nu'$ to $\nu$ with $a(\nu')=a$\\
	$\bsy{s}\leftarrow S(\bsy{s},a)$
	}{\textit{ } \\ \textbf{return} $\nu'$, $\bsy{s}$}
	\\
	\vspace{0.5cm}
\Fn{\textsc{Bestchild}($\nu$,$C_p$,$\rho,U_m$)}{
	$N_{\nu}$ $\leftarrow$ sum($\nu.p.\bsy{Q}[\bsy{D}[\nu.o]]$)\\
	$U$ $\leftarrow$ $[]$\\
	\For{$\nu_c$ in $\nu.\bsy{c}$}{
	$i$ $\leftarrow$ $\bsy{D}[\nu_c.o]$\\
	$N_c$ $\leftarrow$ sum($\nu.\bsy{Q}[i]$)\\
	$e$ $\leftarrow$ $C_p\sqrt{\frac{2\log(N_{\nu})}{N_c}}$\\
	$v$ $ \leftarrow$ \textsc{GetValueW}($\nu_c.\bsy{Q}$)\\
	append $\rho(v+e)+(1-\rho)U_m[\nu_c]$ to $U$
	}
	$\nu_b$ $\leftarrow$ $\nu.\bsy{c}[$argmax($U$)$]$\\
	}{\vspace{0.2cm} \textbf{return} $\nu_b$}
	\\
	\vspace{0.5cm}
\Fn{\textsc{DefaultPolicy}($\nu$, $\bsy{s}$)}{
	\While{$\bsy{s}$ \emph{is nonterminal}}{
	$D$, $\theta \leftarrow G_1(\bsy{x},\bsy{x_d})$ with $\bsy{x}$ the position of $\bsy{s}$\\
	$\bsy{s} \leftarrow S(\bsy{s},\theta)$}
	}{\textit{ } \\ \textbf{return} reward for state $\bsy{s}$}
\\ 
\vspace{0.5cm}
\Fn{\textsc{Backup}($\nu$,$\Delta$)}{
 \While{$\nu.p$ \emph{is not null}}{
 	$\nu.p.\bsy{Q}[\bsy{D}[\nu.o]]$.add($\Delta$)\\
 	$\nu \leftarrow$ $\nu.p$}}{} 
\caption{The UCT algorithm of a WorkerTree}
\end{algorithm} 
\bibliography{references}
\bibliographystyle{aaai}
\end{document}\grid
